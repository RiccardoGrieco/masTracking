%		** COMPORTAMENTO DEGLI AGENTI **


Ogni agente viene inizializzato per mezzo di una serie di \textit{percepts}, i quali vengono estratti da un file JSON di configurazione. Gli agenti telecamera hanno posizione fissa e una porzione di spazio a loro visibile. Tale area di visione pu� essere sovrapposta con quella di altri agenti, ma nessuno degli agenti ha informazioni sui parametri degli altri. Fra i percepts iniziali troviamo anche quello riferito al numero di vicini, cio� il numero di agenti con cui pu� comunicare. Per comodit� di utilizzo del framework di simulazione, questi percepts iniziali diventano \textit{mental notes} dell'agente.

Ogni agente, inoltre, tiene memoria di un intero progressivo con il quale etichetta i target che scopre. I target, infatti, vengono identificati dagli agenti attraverso due attributi: nome dell'agente che lo ha scoperto, un numero progressivo che indica il numero di target che quel particolare agente ha scoperto. Si dice che un agente scopre un target quando lo percepisce nella sua area di visione e, dopo aver chiesto ai suoi vicini se stanno tracciando un target in quella posizione, ottiene solo risposte negative.

Il valore della puntata di un'asta viene specificato attraverso la seguente funzione.\\
Siano $A$ l'agente che deve effettuare la puntata per il target $t_n$, $t$ l'eventuale target che $A$ sta gi� tracciando, $N_A$ il numero di vicini di $A$, $dist$ una funzione distanza (ad es. la distanza Euclidea) fra un agente ed il target che sta tracciando e $k \in \mathbb{N}$ tale che $k >> dist(A, t)$, allora:
\begin{center}
$$
b(A) =
\begin{cases} 	
\frac{(2 \cdot dist(A,t) - dist(A, t_n)) - k}{N_A} & \mbox{se }A \mbox{ sta tracciando }t \\ \\
\frac{\frac{1}{dist(A,t)} + k}{N_A} & \mbox{ altrimenti.}
\end{cases} 
$$
\end{center}
La seguente tabella riassume le \textit{belief} degli agenti.

\begin{tabularx}{\textwidth}{| l | X |}
	\hline
	\textbf{Belief}						&	\textbf{Descrizione} \\ \hline
	numberOfAgents(N)					&	Il numero di agenti totali � N. \\ \hline
	noNeighbor(N)						&	L'agente condivide aree di visione con N altri agenti. \\ \hline
	myPosition(X,Y)						&	La posizione dell'agente nell'ambiente � (X,Y).\\ \hline
	canSee(A,B,C,D)						&	L'ambiente pu� vedere il rettangolo specificato dai punti (A,B) e (C,D). \\ \hline
	progressiNo(N)						&	Numero progressivo dell'agente per l'etichettature di un nuovo target. \\ \hline
	losingTarget(Ag,Tid,X,Y)			&	L'agente sta perdendo il target identificato con (Ag,Tid) nella posizione (X,Y). \\ \hline
	auctionOnGoing(Ag,Tid,X,Y)			&	L'asta per il target (Ag,Tid) che ha bandito l'agente � ancora in corso. \\ \hline
	numberOfPartecipants(Ag,Tid,N)		&	Il numero di partecipante all'asta per il target (Ag,Tid) � N. \\ \hline
	partecipate(Ag,Tid,V)[source(S)]	&	V indica se l'agente S partecipa all'asta per il target (Ag,Tid). \\ \hline
	bid(Ag,Tid,V)[source(S)]			&	V indica la puntata effettuata dall'agente S nell'asta per il target (Ag,Tid). \\ \hline
	winner(Ag,Tid,X,Y)					&	Indica che l'agente � il vincitore dell'asta per il target (Ag,Tid). \\ \hline
	target(X,Y)							&	L'agente ha scoperto un nuovo target in posizione (X,Y). \\ \hline
	tracking(Ag,Tid,X,Y)				&	L'agente sta tracciando il target (Ag,Tid) in posizione (X,Y). \\ \hline
	alreadyTracking(X,Y,T,I)[source(S)]	&	T indica se l'agente S sta gi� tracciando il target in posizione (X,Y). I indica se l'agente S � interessato al target in posizione (X,Y). \\ \hline
\end{tabularx}

La seguente tabella riassume le \textit{regole} degli agenti.

\begin{tabularx}{\textwidth}{|l |X |}
	\hline
	\textbf{Regola}		&	\textbf{Descrizione}	\\ \hline
	amInterested(X,Y)	&	Vero se l'agente � interessato ad un target in posizione (X,Y). \\ \hline
\end{tabularx}

La seguente tabella riassume i \textit{goal} degli agenti.

\begin{tabularx}{\textwidth}{|l |X |}
	\hline
	\textbf{Goal}					&	\textbf{Descrizione}	\\ \hline
	findWinner(Ag,Tid)				&	Calcola il vincitore di un'asta tra i partecipanti rimanenti. \\ \hline
	confirm(Ag,Tid,C)[source(S)]	&	In base alla conferma/rifiuto di vittoria dell'agente S, indicata da C, agisce come da protocollo d'asta. \\ \hline
	clearAuction(Ag,Tid)			&	Elimina i beliefs relativi all'asta per il target (Ag,Tid) dalla belief base dell'agente. \\ \hline
	cfp(Ag,Tid,X,Y)[source(S)]		&	Dichiara l'intenzione dell'agente di partecipare all'asta per il target (Ag,Tid) in posizione (X,Y). \\ \hline
	placeBid(Ag,Tid,X,Y)[source(S)]	&	Calcola la puntata per l'asta per il target specificato e la invia al banditore S. \\ \hline
	tellMeTracking(X,Y)[source(S)]	&	Comunica a S se l'agente sta attualmente tracciando il target in posizione (X,Y). \\ \hline
\end{tabularx}


% perch� non askAll o askOne? non posso gestire il caso in cui tutti rispondono negativamente 