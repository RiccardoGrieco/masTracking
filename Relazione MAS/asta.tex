%		** MECCANISMO D'ASTA **

Le aste rappresentano l'elemento fondamentale dell'interazione fra gli agenti. Tramite le aste, gli agenti possono cedere e acquisire nuovi target da tracciare. Ci sono due momenti in cui un agente pu� bandire un'asta:
\begin{itemize}
	\item quando un target attualmente tracciato � in procinto di abbandonare la sua area di visione
	\item quando, dopo aver ricevuto notifica sulla vincita di un'asta precedente, cerca di cedere il target che gi� sta tracciando.
\end{itemize}
Il protocollo d'asta implementato prevede l'utilizzo di aste \textit{First-Price Sealed-Bid}: un tipo d'asta \textit{one-shot} a busta chiusa. 

Il valore della puntata di un'asta viene specificato attraverso la seguente funzione, la quale � implementata attraverso l'azione interna \textit{calculateBid}.\\
Siano $A$ l'agente che deve effettuare la puntata per il target $t_n$, $t$ l'eventuale target che $A$ sta gi� tracciando, $N_A$ il numero di vicini di $A$, $dist$ una funzione distanza (ad es. la distanza Euclidea) fra un agente ed il target che sta tracciando e $k \in \mathbb{N}$ tale che $k \gg dist(A, t)$, allora:
\begin{center}
	$$
	b(A) =
	\begin{cases} 	
	\frac{(2 \cdot dist(A,t) - dist(A, t_n)) - k}{N_A} & \mbox{se }A \mbox{ sta tracciando }t \\ \\
	\frac{dist(A,t)^{-1} + k}{N_A} & \mbox{ altrimenti.}
	\end{cases} 
	$$
\end{center}
La funzione cos� definita soddisfa i seguenti vincoli:
\begin{itemize}
	\item gli agenti "liberi" vincono sempre sugli agenti "occupati"
	\item fra agenti "liberi" vince sempre il pi� vicino al target
	\item fra agenti "occupati" vince sempre il pi� vicino al target con target attualmente tracciato pi� lontano
	\item a parit� di tutto, vince chi ha meno vicini.
\end{itemize}

Una tipica interazione fra agenti in un contesto d'asta � rappresentata dal seguente \textit{sequence diagram}.

%TODO inserire sequence dell'asta

