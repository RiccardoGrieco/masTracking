L'ambiente di simulazione, rappresentato dalla classe \textit{HouseEnv}, costituisce il fulcro dell'applicativo: attraverso i metodi di questa classe vengono inizializzate le belief base degli agenti e permette di far percepire agli agenti gli spostamenti dei target.
Nel dettaglio permette di:
\begin{itemize}
\item Caricare da file la posizione di ogni agente inizializzando la rispettiva classe java e belief base
\item Stabilire per ogni agente l'area di visione e il numero di agenti che condividono una parte dell'area di visione
\item Tenere traccia delle associazioni agente-target tracciato
\end{itemize}
L'aggiornamento delle belief base degli agenti � stato implementato attraverso un thread invocato ad ogni spostamento dei target che dapprima procede ad eliminare ogni percept per ogni agente e successivamente computa, a partire da uno snapshot del model, i nuovi percept da inserire nelle belief base.
