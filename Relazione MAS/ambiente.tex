L'ambiente di simulazione, rappresentato dalla classe \textit{HouseEnv}, costituisce il fulcro dell'applicazione: i metodi di questa classe permettono di inizializzare le belief base degli agenti e di far percepire a questi gli spostamenti dei target.
Nel dettaglio permette di:
\begin{itemize}
\item caricare da file la posizione di ogni agente inizializzando la rispettiva classe java e belief base
\item stabilire per ogni agente l'area di visione e il numero di agenti che condividono una parte di questa
\item tenere traccia delle associazioni agente-target tracciato
\end{itemize}
L'aggiornamento delle belief base degli agenti � stato implementato attraverso un thread invocato ad ogni spostamento dei target che, dapprima procede ad eliminare i percept di ogni agente e, successivamente, definisce, a partire dalla configurazione del model, i nuovi percept da inserire nelle belief base.