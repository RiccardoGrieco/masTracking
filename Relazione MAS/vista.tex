La vista dell'applicativo � rappresentata dalla classe \textit{HouseView} che estende GridWorldView. Quest'ultima, definita all'interno del package java associato a Jason, prevede che l'environment  e dunque lo spazio di simulazione sia discretizzato attraverso una griglia di posizioni. In particolare permette di disegnare in una griglia di posizioni attraverso \textit{l'override} del metodo \texttt{public void draw(Graphics g, int x, int y, int object)} di parametri:


\begin{itemize}
	\item \textit{Graphics g}: riferimento all'area del Canvas da disegnare
	\item \textit{int x}: ascissa dell'area g. � utilizzata insieme alla lunghezza di ogni area per definire quanta porzione di g disegnare orizzontalmente
	\item \textit{int y}: ordinata dell'area g. � utilizzata insieme all'altezza di ogni area per definire quanta porzione di g disegnare verticalmente
	\item \textit{int object}: Specifica la tipologia dell'oggetto da disegnare. L'idea � quella utilizzare uno switch su object per individuare la giusta texture da disegnare
\end{itemize}
L'implementazione standard di Jason prevede ad ogni cambiamento di posizione di ridisegnare l`intera area Canvas: per evitare fastidiosi flickering dovuti a spostamenti continui dei target si � scelto di avere memoria delle vecchie posizioni occupate e ridisegnare unicamente l'area precedente e la nuova.
