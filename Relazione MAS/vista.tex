La vista dell'applicazione � rappresentata dalla classe \textit{HouseView} che estende GridWorldView. Quest'ultima, definita all'interno del package java associato a Jason, prevede che l'ambiente, dunque lo spazio di simulazione, sia discretizzato attraverso una griglia di posizioni. In particolare permette di disegnare in una griglia di posizioni attraverso l'override del metodo \texttt{public void draw(Graphics g, int x, int y, int object)}.

L'implementazione standard di Jason prevede ad ogni cambiamento di posizione di ridisegnare l'intera area Canvas: per evitare fastidiosi flickering dovuti a spostamenti continui dei target si � scelto di avere memoria delle vecchie posizioni occupate e ridisegnare unicamente l'area precedente e la nuova.
